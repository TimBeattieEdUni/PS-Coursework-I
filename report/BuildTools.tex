% Description of the build tools you used and your views on the strengths and weaknesses of these
\section{Build Tools}
\label{Build Tools}
The project's build system was implemented using GNU Make since several group members were already familiar with this tool.  The project's source code was separated into three directories: \textbf{classes}, \textbf{app}, and \textbf{test}, which contained source code for the project's C++ classes, main application program, and unit test program, respectively.  Separate directories were also created for the project's Doxygen documentation and for the group report in LaTeX format.  A set of recursive Makefiles was set up to perform the following tasks:

\begin{itemize}

  \item Build the C++ classes into a static library.
  \item Build the application, linking in the class library.
  \item Unpack and build the unit test framework (see Section~\ref{Testing}).
  \item Build the unit test program, linking in the class library.
  \item Run the unit test program.
  \item Compile the group report LaTeX source into a PDF document.
  \item Generate Doxygen documentation for all source code.

\end{itemize}

For an explanation of the reasons for using a static library, see Section~\ref{Testing}.  

The main goals of the build system were to save programmer time and reduce errors, and to this end every task which could reasonably be automated was included in the build.  Where possible, dependencies between tasks were implemented.  For example, the \textbf{test} target was made to depend on the \textbf{classes} directory and the UnitTest++ library, ensuring that these were present and up-to-date before building and running the unit test program.  As another example, the \textbf{doc} target was made to depend on all three directories which contained C++ source code, ensuring that the Doxygen generated documentation would be rebuilt if any source had been modified.

Common settings for C++ compilation and linking were placed in a top-level Make include file which was included by the Makefiles in each of the three directories which contained source code.  These settings included directory and file names and the selection of compiler, linker, and library archiver with their command-line options.  This enabled these settings to be maintained in one place for all source code.  

The default target in the top-level Makefile was \textbf{release}, which performed all tasks in the list above.  Following a common pattern for Makefiles, a second \textbf{debug} target was added which differed from \textbf{release} only in its compiler settings: release builds were optimised with the \texttt{-O3} compiler flag, and debug builds were compiled with debug symbols using \texttt{-g} and had the \texttt{\_DEBUG} macro defined with \texttt{-D\_DEBUG}.  The debug build was used only briefly when diagnosing a subtle bug in class Landscape, but on that one occasion it was extremely useful.  (See Section~\ref{Debugging}.)

For profiling the application, a third top-level target was added: \texttt{profile}.  This target ran \texttt{make clean} on the directories \texttt{app} and \texttt{classes} and rebuilt both with profiling enabled using the \texttt{-pg} compiler flag.  To build the application again without profiling support, it was first necessary to run \texttt{make clean} manually either at the top level or in both the \texttt{app} and \texttt{classes} directories.  As with all manual tasks, it would have been beneficial to automate this, but the group was not able to automate triggering \texttt{make} to clean and rebuild the application, classes, and unit test if and only if the previous build had been a profiling build in the time available.

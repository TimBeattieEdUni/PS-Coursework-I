% Ideas for further work or what you would have done if you had more time.
\section{Further Work}
\label{Further Work}

\subsection{Further Work on the Implementation}
Given further time we would have liked to have included a unit test for our timing methods for completeness.
While we have written a function called \texttt{ApplyPopulation} within the \texttt{Landscape} class that would allow the passing of a tailored array of pumas and hares to the class, we ran out of time to implement this as an option within the program. 

\subsection{Further Performance Testing}

We would have liked to have performed more tests with different initialisations of hares and pumas on further realistic maps. For example we would have liked to have input a land/water mask that represented two sides of a river with some bridging points; in this case we would have initialised a large number of hares and pumas on each side of the bridge and investigated how the populations evolved with time.   

As described in Section \ref{Debugging} we experienced bugs that were related to how our program dealt with non-square maps. We were confident however that our code was accurate for square maps so we were able to proceed with performance testing for these cases. 
Given more time we would have liked to have investigated the performance of the code for non-square input maps.
Specifically we would have investigated program duraton while varying one dimension and keeping the other constant. Another interesting test would have been to produce two alternative land/water mask files that were simply rotated ninety degrees and investigate any performance degredation dependant on which dimension is largest. 



 

% courseword.tex 
% 
% MSc in High Performance Computing
%
% Programming Skills Coursework (in C++)
%
% Tim Beattie, David Canning, Marton Feigl, Max Sorantin
%

\title{Programming Skills Coursework}
\author{Tim Beattie, David Canning, Marton Feigl, Max Sorantin}
\date{\today}

%
% Preamble
%

\documentclass[12pt,a4paper]{article}

\usepackage{graphicx}
\usepackage[section]{placeins}
\usepackage{amsmath}
\usepackage{pdfpages}
\usepackage{a4wide}


\usepackage{parskip}
\setlength{\parindent}{0pt}
\setlength{\parskip}{10pt plus 1pt minus 1pt}


%
% The following defines an environment for including source with syntax hilighting.
% (Copied from stackoverflow.com/questions/3175105/how-to-insert-code-%into-a-latex-doc) 
% Could be useful if we want to include source code in the report. 
%
% To use a different language, overwrite the language paramter in the code. I.e. write: 
%	\lstset{language=bash}
% before the beginning of the listing. 

% Then enter your code: 
%	\begin{lstlisting}
%		source code here ...
%	\end{lstlisting}
%

\usepackage{listings}
\usepackage{color}

\definecolor{dkgreen}{rgb}{0,0.6,0}
\definecolor{gray}{rgb}{0.5,0.5,0.5}
\definecolor{mauve}{rgb}{0.58,0,0.82}

\lstset{frame=tb,
  language=c++,
  aboveskip=3mm,
  belowskip=3mm,
  showstringspaces=false,
  columns=flexible,
  basicstyle={\small\ttfamily},
  numbers=none,
  numberstyle=\tiny\color{gray},
  keywordstyle=\color{blue},
  commentstyle=\color{dkgreen},
  stringstyle=\color{mauve},
  breaklines=true,
  breakatwhitespace=true,   
  tabsize=3
}


%Start of the Document Proper

\begin{document}

%Create Title Page
\maketitle
\newpage

%Number pages with Contents, Figure Table, etc. in Roman Numerals.
\pagenumbering{roman}

\tableofcontents
%\listoffigures 	
\newpage

%Begin normal page numbering from first section. 
\pagenumbering{arabic}



%include all the sections
% Short Introduction
\section{Introduction}
\label{Introduction}
This report describes a group project to develop a program which simulated puma and hare populations in a landscape.  
The software development process from planning to completion is described as is the experience of working in a group and how this was facilitated by the appropriate use of software tools and techniques.

The project was submitted as a compressed tar file, "popsim.tar.gz" which can be unpacked with the following command:

\begin{lstlisting}
	tar -xvzf popsim.tar.gz
\end{lstlisting}

Once the tar file is unpacked in the current directory the project can be built with:

\begin{lstlisting}
	make
\end{lstlisting}

This will build the program, build and run the project's unit tests, generate Doxygen documentation for all source code in the project, and compile this report document from LaTeX source.  The program is called "popsim" and can be found in the directory "app" in the directory where the tar file was unzipped.  An example configuration file and landscape bitmask are also included in the "app" directory.  The program can be run using these files as follows:

\begin{lstlisting}
	cd app
	./popsim popsim.cfg small.dat
\end{lstlisting}

Running the program without any arguments will print a short "usage" message.

Documentation generated from the project's source code can be found in the following file:

\begin{lstlisting}
	doc/doxygen/index.html
\end{lstlisting}

There follows a description of each stage of the project's development.

% How you planned your work
\section{Planning}
\label{Planning}

At the group's first meeting it was decided to create a task list with each member of the group suggesting tasks which would be invoved in completing the project.  Each task was then assigned one of the following priorities:

\begin{itemize}
	\item \textit{Essential:} The task must be done in order to consider the project complete.
	\item \textit{Desirable:} Completing the task was not essential but would likely improve the group's mark for the coursework.
	\item \textit{Polish:} The task was above and beyond the requirements for the project.
\end{itemize}

The group agreed to work on only essential tasks until these were all complete, then on desirable tasks, and only then on tasks designated as polish.

Decisions were then made concerning the infrastructure required for the group to work together: the choice of version control system, build tool, and unit testing framework. The group also agreed on a basic coding standard concerning indentation, placement of curly braces, and file names for C++ code.

The overall program design and layout were then discussed (see sec.(\ref{Design})), focussing on the set of classes which would be required and the information these classes would exchange.

Additional tasks were identified, prioritised, and added to the task list throughout the duration of the project.


% Brief Summary of what tasks each member of the group did
\section{Assignment of tasks to group members}
\label{Assignment of tasks to group members}
% Description of your design
\section{Design}
\label{Design}
% Description of the programming language you used and how useful it was
\section{Programming Language}
\label{Programming Language}
% Description of the revision control ou used
\section{Revision Control}
\label{Revision Control}

The group was unanimous that revision control should be used. Git was chosen, both because several team members had used it before and because GitHub's free service represented an ready and simple option for hosting the "origin" repository.
Subversion was considered for its greater ease of use, but the team decided that Git's ability to function in the absence of a network connection outweighed its steeper learning curve.\\

Using a central repository encouraged the group to make both the code and the group report modular, separating each into as many separate sub-tasks as possible in order to enable any group member to work on any task in isolation without interfering with the work of other group members.
The large number of small tasks made the distribution of work easier as any group member who finished a task sooner than expected could take on the next task which suited their skill set.\\

The build system's one-step build process encouraged the habit of building and testing the entire project after each change before pushing to the central repository, and this kept the quality of the work in the repository high.
The group had a working rule that changes should only be pushed to the Git repository if the following conditions were met:

\begin{itemize}
  \item The application would still build.
  \item The unit tests would still build and pass.
  \item The code had a good level of readability and tidiness.
  \item All classes, functions, and parameters were annotated with Doxygen comments.
\end{itemize}

If a change broke the ability to build the project, the group member who pushed the change was responsible for fixing the repository immediately.
This ensured that the most recent revision could always be worked on by everyone in the group. Every member of the group did break the repository at least once, but fixes were applied quickly, and the ability of the group to work on the project was maintained.\\

A separate "scratch" directory was created in the repository for sharing code and ideas which weren't ready to go into the main project, and this was used to exchange task lists, first drafts of report text, and code about which the writer had questions for the other members of the group.\\

In summary, the Revision Controll system proved to be a powerful and yet easy to use tool as besides the usual benefits of backup and common availability of the recent version, it also led to high standard of tidyness in the repository, due to the imposed rules, as highlighted above.

% Description of the build tools you used and your views on the strengths and weaknesses of these
\section{Build Tools}
\label{Build Tools}
The project's build system was implemented using GNU Make since several group members were already familiar with this tool.  The project's source code was separated into three directories: \textbf{classes}, \textbf{app}, and \textbf{test}, which contained source code for the project's C++ classes, main application program, and unit test program, respectively.  Separate directories were also created for the project's Doxygen documentation and for the group report in LaTeX format.  A set of recursive Makefiles was set up to perform the following tasks:

\begin{itemize}

  \item Build the C++ classes into a static library.
  \item Build the application, linking in the class library.
  \item Unpack and build the unit test framework (see Section~\ref{Testing}).
  \item Build the unit test program, linking in the class library.
  \item Run the unit test program.
  \item Compile the group report LaTeX source into a PDF document.
  \item Generate Doxygen documentation for all source code.

\end{itemize}

For an explanation of the reasons for using a static library, see Section~\ref{Testing}.  

The main goals of the build system were to save programmer time and reduce errors, and to this end every task which could reasonably be automated was included in the build.  Where possible, dependencies between tasks were implemented.  For example, the \textbf{test} target was made to depend on the \textbf{classes} directory and the UnitTest++ library, ensuring that these were present and up-to-date before building and running the unit test program.  As another example, the \textbf{doc} target was made to depend on all three directories which contained C++ source code, ensuring that the Doxygen generated documentation would be rebuilt if any source had been modified.

Common settings for C++ compilation and linking were placed in a top-level Make include file which was included by the Makefiles in each of the three directories which contained source code.  These settings included directory and file names and the selection of compiler, linker, and library archiver with their command-line options.  This enabled these settings to be maintained in one place for all source code.  

The default target in the top-level Makefile was \textbf{release}, which performed all tasks in the list above.  Following a common pattern for Makefiles, a second \textbf{debug} target was added which differed from \textbf{release} only in its compiler settings: release builds were optimised with the \texttt{-O3} compiler flag, and debug builds were compiled with debug symbols using \texttt{-g} and had the \texttt{\_DEBUG} macro defined with \texttt{-D\_DEBUG}.  The debug build was used only briefly when diagnosing a subtle bug in class Landscape, but on that one occasion it was extremely useful.  (See Section~\ref{Debugging}.)

For profiling the application, a third top-level target was added: \texttt{profile}.  This target ran \texttt{make clean} on the directories \texttt{app} and \texttt{classes} and rebuilt both with profiling enabled using the \texttt{-pg} compiler flag.  To build the application again without profiling support, it was first necessary to run \texttt{make clean} manually either at the top level or in both the \texttt{app} and \texttt{classes} directories.  As with all manual tasks, it would have been beneficial to automate this, but the group was not able to automate triggering \texttt{make} to clean and rebuild the application, classes, and unit test if and only if the previous build had been a profiling build in the time available.

% Description of what testing you did and any test frameworks that were used
\section{Testing}
\label{Testing}

The project's unit testing was done with UnitTest++, available at the following website:
\begin{center}
 http://unittest-cpp.sourceforge.net
\end{center}
From this page:
\begin{center}
``UnitTest++ is a lightweight unit testing framework for C++.  It was designed to do test-driven development on a wide variety of platforms. Simplicity, portability, speed, and small footprint are all very important aspects of UnitTest++.''
\end{center}

UnitTest++ was chosen because it was freely available, cross-platform, simple to set up and use, and equipped with a small but effective set of unit testing features.
%Documentation for UnitTest++ can be found either on the UnitTest++ home page or in the following file in the project (once the project has been built):
%UnitTest++/docs/UnitTest++.html
\subsection{Implementation of the unittest framework}
UnitTest++ was downloaded as a .zip file and added to the project's Git repository in the "downloads" directory.
UnitTest++ was then integrated into the project's build system.
Running "make test" in the project's top-level directory will unzip the downloaded file to the directory "UnitTest++", call UnitTest++'s makefile to build the UnitTest++ static library, run its self-tests and build plus run the project's unit tests which link in that library.

To ensure that the binary code being unit tested was the same code included in the popsim program, the project's build system was configured to build all C++ classes into a static library which was then linked into both the popsim program and the unit test program. This avoided problems which can occur when compiler or linker settings differ between multiple builds of the same source code.
The project's unit test program was implemented in a small C++ implementation file which provided main() and included test headers for the project's C++ classes.  Each class's test header contained one or more UnitTest++ "TEST" macros which implemented unit tests for that class.  The $main()$ function simply contained a call to $UnitTest::RunAllTests()$. All test code was placed in the "test" directory.

\subsection{Unit tests for specific classes}
The project was begun with the firm intention of writing each class's unit tests during the development of that class, ensuring that every feature of every class would be robust and dependable before being used by the popsim program.  However, as the deadline approached, pressure was felt and several key classes were written without tests, with the intention to write tests later.
As a result the program had a bug in the ordering of pixels in PPM and bitmask files and cells in the landscape array: pixels in files were written by iterating over rows before columns, but elements in the array were being accessed by iterating over columns first, then rows.  The bug was present in several different classes, all of which did not yet have unit tests.
The bug - and the loss of a day - could have been prevented by writing even basic unit tests of the affected classes before using them in the program.



% Description of hw you did any debugging any any tools that were used       
\section{Debugging}
\label{Debugging}
The most coding errors where picked up by the compiler and therefore the need of a debugging tool, such as "gdb", was furtunatelly very limited.
However, it proved to be essential in finding a a 
The "gdb" debugger was only used infrequently to ensure correct flow through of the program. 
% Description of what testing you did and any test frameworks that were usedi
\section{Performance Tests and Analysis}
\label{Performance Tests and Analysis}

Timing and profiling runs were performed on the CP Lab machines due to the availability of a working version of gprof, which was not present on all of the group's own computers.
Optimisation was performed in two stages: changes to the code, and changes to the compiler optimisation flags.  For the first stage the $-O3$ flag was used for all runs.  The second stage chose the best-performing version of the program from the first stage and observed the effects of compiling with the $-O4$ and $-O5$ flags.
Run time of the program was measured over a series of square landscapes with sizes ranging from 200x200 to 2000x2000.
Profiling runs were performed using an 800x800 landscape.
All timing runs were configured to perform 1250 time steps with PPM file output every 100 time steps.
Program speedup \textit{S} for subsequent runs was calculated with the following equation:

\begin{equation} 
S = T_{ref} / T 
\label{equation:speedup}
\end{equation}

...where $T_{ref}$  was the time taken by the initial reference run and ${T}$ was the time taken by the run for which speedup was being calculated.


PUT 1ST GRAPH HERE?


\subsection{Code optimisation}
\label{Code optimisation}

\begin{table}[h!]
\caption{Tabulated partial output from gprof on first profiled version of popsim.}
\label{tab:profile1}
\begin{center}
\begin{tabular}{|c|c|c|c|c|}
\hline
time [\%] & time [s] & self time [s] & calls & name\\
\hline
47.00 & 58.77 & 58.77 & 1250 & $Landscape::Update()$\\
\hline
24.30 & 89.15 & 30.38& 4806406416 & $Array2D::operator()$\\
\hline
10.75& 102.59 & 13.44 & 511439120 & $std::vector::operator[]$\\
\hline
\end{tabular}
\end{center}
\end{table}

Table \ref{tab:profile1} shows partial output from gprof for the initial version of the code.
Three functions accounted for approximately 82\% of the run time, with nearly half of the run time spent in $Landscape::Update()$.
This was to be expected as the bulk of the program's work was done in this function.
However, nearly one quarter of the run time was spent accessing elements of the landscape array, and this suggested a simple modification: removing the bounds checking on every element access in $Array2D<T>::operator()$. 
This change was made and a second timing run and profiling run were done.

\begin{table}[h!]
\caption{Tabulated partial output from gprof after array bounds checking removal.}
\label{tab:profile2}
\begin{center}
\begin{tabular}{|c|c|c|c|c|}
\hline
time [\%] & time [s] & sef time [s] & calls & name\\
\hline
50.30 & 58.25 & 58.25 & 1250 & $Landscape::Update()$\\
\hline
15.18 & 75.83 & 17.58 & 4806406416 & $Array2D::operator()$\\
\hline
13.72 & 91.71 & 15.89 & 511439120 & $std::vector::operator[]$\\
\hline
\end{tabular}
\end{center}
\end{table}

Table \ref{tab:profile2} shows partial output from gprof for the version of the code with array bounds removed.  
The time spent in $Array2D::operator()$ was significantly reduced compared to the first profiling run.  
While bounds checking was included in the initial version of $Array2D<T>::operator()$ for safety, once the program had been demonstrated to work correctly, the considerable performance improvement achieved by removing bounds checking was considered to be worthwhile.
The reference documentation for class $std::vector$ states that element access with bounds checking is provided by the function $std::vector<T>::at()$.
A timing run was performed with $std::vector<T>::operator[]$ replaced with $std::vector<T>::at()$, resulting in a run time of 45.721 seconds, representing a program speedup of 1.28.
This was an improvement but still considerably slower than with no bounds checking at all. 
All three options were left in the code in $Landscape::Update()$ with the unused options commented out.

A second optimisation became apparent on examination of the for loops in $Landscape::Update()$.
The code was iterating over the x-coordinate of a cell, then over its y-coordinate.
While this was the intuitive order, it caused the code to stride through the array of cells by a distance proportional to the width of the landscape array instead of accessing the next cell in memory each time.
The order of the iterations was swapped, and another profiling and timing run were performed.
Run time with the order of iterations swapped was 36.48 seconds, representing a program speedup of 1.60.

\begin{table}[h!]
\caption{Tabulated partial output from gprof after iteration order swap.}
\label{tab: profile 3}
 \begin{center}
\begin{tabular}{|c|c|c|c|c|}
\hline
name & time [\%] & time [s] & sef time [s] & calls\\
\hline
$Landscape::Update()$ & 48.54 & 59.65 & 59.65 &1250 \\
\hline
$Array2D::operator()$& 17.45& 81.10& 21.45 &4806406416 \\
\hline
$std::vector::operator[]$& 12.34& 96.27 &15.17 &511439120\\
\hline
\end{tabular}
\end{center}
\end{table}

Table \ref{tab:profile3} shows partial output from gprof for the version of the code with array bounds removed and with iteration order reversed.
This version spent a slightly lower percentage of its run time in function Landscape::Update().
The improvement in performance was less than expected given that memory loading into cache was theoretically optimised.

A further optimisation was considered but not implemented due to time constraints: storing pointers to cell neighbours in each cell so that function $Landscape::Update()$ could access each cell's neighbours without calling $Array2D::operator()$ four times.
These pointers would be initialised at program startup meaning the array accesses would be done just once instead of once per update and reducing the number of memory accesses to non-local parts of the array.
However, there was insufficient time for this change to be implemented and tested.\\

The group also considered checking whether any difference in performance could be found between runs on landscapes with different arrangements of land and water due to branch prediction as the Landscape class verifies that cell is land before performing the calculation for that cell.
A landscape consisting entirely of islands of one cell would presumably cause the branch prediction to be wrong most if not all of the time, which would presumably reduce performance compared to a landscape with the same number of land cells arranged at the top or bottom of the landscape, corresponding to the beginning or end of the landscape array.


\subsection{Compiler optimisation.}
\label{Compiler optimisation}

With array bounds checking removed and iteration order swapped, two more versions of the program were built with the compiler optimisation flag set to $-O4$ and $-O5$.
Timing runs were then performed and are compared to the version compiled with $-O3$ in graph REFERENCE HERE.

It was found that the various compiler optimisation flags made only a slight improvement for some landscape sizes and actually reduced performance in some cases.


\begin{table}
\caption{Timing for increasing landscape sizes}
\label{tab: Size timing}
 \begin{center}
\begin{tabular}{|c|c|c|}
\hline
Ladscape Size & run time [s] & delta run time[s]\\
\hline
200 & 1.8722 & ×\\
\hline
400 & 11.662 & 9.7898\\
\hline
600 & 25.526 & 13.864\\
\hline
800 & 48.755 & 23.229\\
\hline
1000 & 81.310 & 32.555\\
\hline
1200 & 97.364 & 16.054\\
\hline
1400 & 145.020 & 47.656\\
\hline
1600 & 220.000 & 74.98\\
\hline
1800 & 277.366 & 57.366\\
\hline
200 & 346.391 & 69.025\\
\hline
\end{tabular}
\end{center}
\end{table}





% Some brief conclusions 
\section{Conclusions}
\label{Conclusions}

The project did not reach the stage where any serious investigation could be performed into the behaviour of populations over time given various starting conditions.  
However, the program's behaviour was sanity-tested by starting with a low density of hares and verifying by visual inspection of the produced .ppm files that the pumas died out and the hare population increased afterwards, but the group did not have time to include this in this report.

The group found that splitting a large task into many small tasks was an efficient way to share work between members.  
This process benefitted greatly from a modular program design which could be divided in this way and from using a programming language designed for modularity and code reuse.  
Using revision control facilitated both distribution of completed work and communication about ideas and work in progress. 
Careful control of what was allowed to enter the repository, enabled by a simple, one-step build and test of all deliverables, ensured that the group always had a known-good product which could be incrementally improved throughout the time available.

The group also found that clear and frequent communication was essential when working in a team. 

Finally, the group learned from first-hand experience that skipping unit testing, regardless of deadline pressure, can cost significantly more time than it saves.


% Ideas for further work or what you would have done if you had more time.
\section{Further Work}
\label{Further Work}

\subsection{Further Work on the Implementation}
Given further time the group would have liked to include a unit test for the timing methods for completeness.
While a function called \texttt{ApplyPopulation} was added to the \texttt{Landscape} class which would allow the passing of a tailored array of pumas and hares to the class, there was insufficient time to implement this as an option within the program. 

\subsection{Further Performance Testing}

As described in Section \ref{Debugging} there was an array-indexing bug which was not apparent until the program was run on a non-square landscape. 
Given more time the group would have liked to investigate the performance of the code for non-square input maps.
Specifically the group could have investigated program duraton while varying one dimension and keeping the other constant. 
Another interesting test would have been to produce two alternative land/water mask files that were simply rotated ninety degrees and investigate any performance degredation dependant on which dimension is largest. 



 




\end{document}

% courseword.tex 
% 
% MSc in High Performance Computing
%
% Programming Skills Coursework (in C++)
%
% Tim Beattie, David Canning, Marton Feigl, Max Sorantin
%

\title{Programming Skills Coursework}
\author{Tim Beattie, David Canning, Marton Feigl, Max Sorantin}
\date{\today}

%
% Preamble
%

\documentclass[12pt,a4paper]{article}

\usepackage{graphicx}
\usepackage[section]{placeins}
\usepackage{amsmath}

%
% The following defines an environment for including source with syntax hilighting.
% (Copied from stackoverflow.com/questions/3175105/how-to-insert-code-%into-a-latex-doc) 
% Could be useful if we want to include source code in the report. 
%
% To use a different language, overwrite the language paramter in the code. I.e. write: 
%	\lstset{language=bash}
% before the beginning of the listing. 

% Then enter your code: 
%	\begin{lstlisting}
%		source code here ...
%	\end{lstlisting}
%

\usepackage{listings}
\usepackage{color}

\definecolor{dkgreen}{rgb}{0,0.6,0}
\definecolor{gray}{rgb}{0.5,0.5,0.5}
\definecolor{mauve}{rgb}{0.58,0,0.82}

\lstset{frame=tb,
  language=c++,
  aboveskip=3mm,
  belowskip=3mm,
  showstringspaces=false,
  columns=flexible,
  basicstyle={\small\ttfamily},
  numbers=none,
  numberstyle=\tiny\color{gray},
  keywordstyle=\color{blue},
  commentstyle=\color{dkgreen},
  stringstyle=\color{mauve},
  breaklines=true,
  breakatwhitespace=true,   
  tabsize=3
}


%Start of the Document Proper

\begin{document}

%Create Title Page
\maketitle
\newpage

%Number pages with Contents, Figure Table, etc. in Roman Numerals.
\pagenumbering{roman}

\tableofcontents
%\listoffigures 	
\newpage

%Begin normal page numbering from first section. 
\pagenumbering{arabic}

%
% Short Introduction
%

\section{Introduction}

Here is an example of what the C++ source would look like using the listing: 
 
\begin{lstlisting}
#include<iostream> 

int main (void)
{
	std::cout << "Hello, world!" << endl;

	return 0; 
}
\end{lstlisting}

Writing equations:  

\begin{equation} 
E_{tot} = m c^2 / \sqrt{1 - {v^2/c^2}}
\label{equation:chunktime}
\end{equation}

We can then refer back to equation \ref{equation:chunktime} like so. 


%
% How you planned your work
%

\section{Planning/summary of workflow}
\label{Planning}
At the first meeting it was decided to set up version control, for details see sec.\ref{Revision Control}, and to create a Tasklist.
The latter was structured along the lines described in the "starting your project" section of the courserwork description and everyone
added specific tasks he could think of to the variouse sections and mark them as essential, i.e. needed for a basic implementation of the project, or polish.
After this Tasklist (todo: include tasklist) was established we agreed on a coding standard, build system and test framework, see sec.(\ref{Build Tools}) and sec.(\ref{Testing}), which was setup subsequentely.
It was than decided on the overall programm design/layout, see sec.(\ref{Design}), which led to writing of the needed code/classes and their Unit Tests.
With the essential part of the Tasklist therefore implemented the focus was shifted to the "polish" tasks starting with profiling and optimisation (see sec.(\ref{Performance Tests and Analysis}).
Due to time limitations, this naturally led to report writing.

%
% Brief Summary of what tasks each member of the group did
%

\section{Assignment of tasks to group members}
\label{Assignment of tasks to group members}

%
% Description of your design
%

\section{Design}
\label{Design}
%
% Description of the programming language you used and how useful it was
%

\section{Programming Language}
\label{Programming Language}
%
% Description of the revision control ou used
%

\section{Revision Control}
\label{Revision Control}
%
% Description of the build tools you used and your views on the strengths and weaknesses of these
%

\section{Build Tools}
\label{Build Tools}
%
% Description of what testing you did and any test frameworks that were usedi
%

\section{Testing}
\label{Testing}
%
% Description of hw you did any debugging any any tools that were used       
%

\section{Debugging}
\label{Debugging}
%
% Description of what testing you did and any test frameworks that were usedi
%

\section{Performance Tests and Analysis}
\label{Performance Tests and Analysis}
%
% Some brief conclusions 
%

\section{Conclusions}
\label{Conclusions}
%
% Ideas for further work or what you would have done if you had more time.
%

\section{Further Work}
\label{Further Work}

\end{document}

% courseword.tex 
% 
% MSc in High Performance Computing
%
% Programming Skills Coursework (in C++)
%
% Tim Beattie, David Canning, Marton Feigl, Max Sorantin
%

\title{Programming Skills Coursework}
\author{Tim Beattie, David Canning, Marton Feigl, Max Sorantin}
\date{\today}

%
% Preamble
%

\documentclass[12pt,a4paper]{article}

\usepackage{graphicx}
\usepackage[section]{placeins}
\usepackage{amsmath}
\usepackage{pdfpages}

\usepackage{parskip}
\setlength{\parindent}{15pt}



%
% The following defines an environment for including source with syntax hilighting.
% (Copied from stackoverflow.com/questions/3175105/how-to-insert-code-%into-a-latex-doc) 
% Could be useful if we want to include source code in the report. 
%
% To use a different language, overwrite the language paramter in the code. I.e. write: 
%	\lstset{language=bash}
% before the beginning of the listing. 

% Then enter your code: 
%	\begin{lstlisting}
%		source code here ...
%	\end{lstlisting}
%

\usepackage{listings}
\usepackage{color}

\definecolor{dkgreen}{rgb}{0,0.6,0}
\definecolor{gray}{rgb}{0.5,0.5,0.5}
\definecolor{mauve}{rgb}{0.58,0,0.82}

\lstset{frame=tb,
  language=c++,
  aboveskip=3mm,
  belowskip=3mm,
  showstringspaces=false,
  columns=flexible,
  basicstyle={\small\ttfamily},
  numbers=none,
  numberstyle=\tiny\color{gray},
  keywordstyle=\color{blue},
  commentstyle=\color{dkgreen},
  stringstyle=\color{mauve},
  breaklines=true,
  breakatwhitespace=true,   
  tabsize=3
}


%Start of the Document Proper

\begin{document}

%Create Title Page
\maketitle
\newpage

%Number pages with Contents, Figure Table, etc. in Roman Numerals.
\pagenumbering{roman}

\tableofcontents
%\listoffigures 	
\newpage

%Begin normal page numbering from first section. 
\pagenumbering{arabic}

Here is an example of what the C++ source would look like using the listing: 
 
\begin{lstlisting}
#include<iostream> 

int main (void)
{
	std::cout << "Hello, world!" << endl;

	return 0; 
}
\end{lstlisting}

Writing equations:  

\begin{equation} 
E_{tot} = m c^2 / \sqrt{1 - {v^2/c^2}}
\label{equation:chunktime}
\end{equation}

We can then refer back to equation \ref{equation:chunktime} like so. 


%include all the sections
% Short Introduction
\section{Introduction}
\label{Introduction}
This report describes a group project to develop a program which simulated puma and hare populations in a landscape.  
The software development process from planning to completion is described as is the experience of working in a group and how this was facilitated by the appropriate use of software tools and techniques.

The project was submitted as a compressed tar file, ``popsim.tar.gzx'' which can be unpacked with the following command:

\begin{lstlisting}
	tar -xvzf popsim.tar.gz
\end{lstlisting}

Once the tar file is unpacked in the current directory the project can be built with:

\begin{lstlisting}
	make
\end{lstlisting}

This will build the program, build and run the project's unit tests, generate Doxygen documentation for all source code in the project, and compile this report document from \LaTeX source.  The program is called \textbf{popsim} and can be found in the directory \textbf{app} in the directory where the tar file was unzipped.  An example configuration file and landscape bitmask are also included in the \textbf{app} directory.  The program can be run using these files as follows:
   

\begin{lstlisting}
	cd app
	./popsim popsim.cfg small.dat
\end{lstlisting}

Running the program without any arguments will print a short ``usage'' message.

Documentation generated from the project's source code can be found in the following file:

\begin{lstlisting}
	doc/doxygen/index.html
\end{lstlisting}

There follows a description of each stage of the project's development.

% How you planned your work
\section{Planning}
\label{Planning}
At the first meeting it was decided to set up version control as detailed in sec.\ref{Revision Control}, and to create a tasklist.

The tasklist was structured along the lines described in the \textit{starting your project} section of the courserwork description; each team member suggested specific tasks relevant to the various sections and these were marked as \textit{essential}, i.e. needed for a basic implementation of the project, or \textit{polish}.

After the tasklist was established we agreed on a coding standard, build system and test framework, see sec.(\ref{Build Tools}) and sec.(\ref{Testing}), which was subsequently setup.
The overall program design and layout was then discussed, see sec.(\ref{Design}), which focussed on the structure of the classes and their Unit Tests and the code required to implement them.
With the essential part of the Tasklist therefore implemented the focus was planned to be shifted to the \tasktt{polish} tasks starting with profiling and optimisation (see sec.(\ref{Performance Tests and Analysis}).

However, in the real execution this was followed only up to the actual coding of the unit tests as profiling and optimisation was performed before careful unit tests were in place. 

% Brief Summary of what tasks each member of the group did
\section{Assignment of tasks to group members}
\label{Assignment of tasks to group members}

Tasks were assigned to group members according to their skills and experience taking into account the need for some tasks, such as setting up revision control and creating a build system, to be done first.  
The group was evenly divided between those with strong skills in C++ and LaTex.  
These were considered equally important due to the division of marks between the code and the report.

Once the essentials were in place, the division of tasks was driven by the task list (see sec.(\ref{Planning})).  
Each group member chose a task which was both high-priority and suited to their skills, informed the group that they were working on that specific task, completed the task, updated the task list, and repeated the process.  
This process kept all group members busy throughout the project's progress.

The resulting division of labour was as follows:

Tim Beattie:
\begin{itemize}
	\item Set up repository on GitHub.
	\item Create build system.
	\item Set up unit testing framework.
	\item Set up Doxygen.
	\item Design of C++ classes.
	\item Implement configuration file reader.
	\item Profiling and optimisation.
	\item Review and tidy C++ source code.
\end{itemize}

David Canning:
\begin{itemize}
	\item Implement bitmask file reader.
	\item Implement PPM file writer.
	\item Create initial LaTex framework for group report.
\end{itemize}


Marton Feigl:
\begin{itemize}
	\item Implement population density calculation.
	\item Implement population average calculation.
\end{itemize}

Max Sorantin:
\begin{itemize}
	\item Implement timing measurement.
	\item Create land/water bitmask files for testing.
	\item Create graphs for group report.
	\item Review and tidy LaTeX source for group report.
\end{itemize}

The top-level design of the program was worked on by the group together.  Each group member wrote text for the group report and reviewed text written by others.


% Description of your design
\section{Design}
\label{Design}
% Description of the programming language you used and how useful it was
\section{Programming Language}
\label{Programming Language}

The project's source code was written in C++ which is a general purpose, object-oriented programming language. 
The group members were all familiar with the language's features, and this made development of the popsim program relatively straightforward. 
C++'s Object Oriented features easily enabled modular programming with program implementation details abstracted away behind interfaces.

As described in Sections ~\ref{Revision Control} and ~\ref{Testing} this modularity lent itself well to a ``many small tasks'' approach to working as a group.  
Each group member was able to work on separate parts of the program and add their work to the Git repository with minimal impact on the work of others.

Using C++ classes to create objects with both names and behaviours enabled a short $main.cpp$ which described what the program should do rather than exposing the details of how it was done.  
For example, the landscape was updated by calling \texttt{Landscape::Update()} and .ppm files were written by calling \texttt{LandscapePpmWriter::Write()}.

Finally, C++'s common RAII idiom meant that very little work had to be done to release resources when the program no longer needed them.  
For example, using a \texttt{std::vector} in class \texttt{Array2D} meant that the dynamically-allocated array was automatically freed from the heap and opened files were automatically closed by class \texttt{std::fstream}. 
No code was required in the destructor of any class written by the group.

% Description of the revision control ou used
\section{Revision Control}
\label{Revision Control}
% Description of the build tools you used and your views on the strengths and weaknesses of these
\section{Build Tools}
\label{Build Tools}
% Description of what testing you did and any test frameworks that were used
\section{Testing}
\label{Testing}

The project's unit testing was done with UnitTest++, available at the following website:
\begin{center}
 http://unittest-cpp.sourceforge.net
\end{center}
From this page:
\begin{center}
``UnitTest++ is a lightweight unit testing framework for C++.  It was designed to do test-driven development on a wide variety of platforms. Simplicity, portability, speed, and small footprint are all very important aspects of UnitTest++.''
\end{center}

UnitTest++ was chosen because it was freely available, cross-platform, simple to set up and use, and equipped with a small but effective set of unit testing features.
%Documentation for UnitTest++ can be found either on the UnitTest++ home page or in the following file in the project (once the project has been built):
%UnitTest++/docs/UnitTest++.html
\subsection{Implementation of the unittest framework}
UnitTest++ was downloaded as a .zip file and added to the project's Git repository in the "downloads" directory.
UnitTest++ was then integrated into the project's build system.
Running "make test" in the project's top-level directory will unzip the downloaded file to the directory "UnitTest++", call UnitTest++'s makefile to build the UnitTest++ static library, run its self-tests and build plus run the project's unit tests which link in that library.

To ensure that the binary code being unit tested was the same code included in the popsim program, the project's build system was configured to build all C++ classes into a static library which was then linked into both the popsim program and the unit test program. This avoided problems which can occur when compiler or linker settings differ between multiple builds of the same source code.
The project's unit test program was implemented in a small C++ implementation file which provided main() and included test headers for the project's C++ classes.  Each class's test header contained one or more UnitTest++ "TEST" macros which implemented unit tests for that class.  The $main()$ function simply contained a call to $UnitTest::RunAllTests()$. All test code was placed in the "test" directory.

\subsection{Unit tests for specific classes}
The project was begun with the firm intention of writing each class's unit tests during the development of that class, ensuring that every feature of every class would be robust and dependable before being used by the popsim program.  However, as the deadline approached, pressure was felt and several key classes were written without tests, with the intention to write tests later.
As a result the program had a bug in the ordering of pixels in PPM and bitmask files and cells in the landscape array: pixels in files were written by iterating over rows before columns, but elements in the array were being accessed by iterating over columns first, then rows.  The bug was present in several different classes, all of which did not yet have unit tests.
The bug - and the loss of a day - could have been prevented by writing even basic unit tests of the affected classes before using them in the program.



% Description of hw you did any debugging any any tools that were used       
\section{Debugging}
\label{Debugging}
% Description of what testing you did and any test frameworks that were usedi
\section{Performance Tests and Analysis}
\label{Performance Tests and Analysis}

\subection{Profiling and optimisation}
\label{Profiling and optimisation}

Profiling runs were done on the CP Lab machines with an all-land 800x800 cell landscape for all runs of the program.  The CP Lab machines were chosen because of the availability of a working version of gprof, which was not present on all of the group's own computers. 
Timing runs were done on a faster machine: a core i7 laptop with a solid-state hard drive. 
This machine was chosen in order to save time as the CP Labs machines were considerably slower. 
It was assumed that since the program's work was primarily calculation and memory access that profiling would produce reliable results even if timings were done on a different machine.\\

For reference an initial run of the program was done and was found to take 47.437 seconds.  Profiling the first implementation of the popsim program revealed that three functions accounted for approximately 80\% of the run time, see tab.(\ref{tab: profile 1}).


\begin{table}
\caption{Output from gprof on 800x800 all-land array on CP Lab}
\label{tab: profile 1}
 \begin{center}
\begin{tabular}{|c|c|c|c|c|}
\hline
name & time [\%] & time [s] & sef time [s] & calls\\
\hline
$Landscape::Update()$ & 47.00 & 58.77 & 58.77 & 1250\\
\hline
$Array2D::operator()$& 24.30 & 89.15 & 30.38& 4806406416\\
\hline
$std::vector::operator[]$& 10.75& 102.59 & 13.44 & 511439120\\
\hline
\end{tabular}
\end{center}
\end{table}


The program spent more time in $Landscape::Update()$ than any other function.
This was to be expected as all of the calculation was done there. However, one quarter of the run time was spent accessing elements of the landscape array, and this suggested a simple modification which improved performance significantly.
The removal of the bounds checking on every element access in $Array2D<T>::operator()$. This change was made and the profiling run was redone, producing the results presented in tab.(\ref{tab: profile 2}).

\begin{table}
\caption{second profile.}
\label{tab: profile 2}
 \begin{center}
\begin{tabular}{|c|c|c|c|c|}
\hline
name & time [\%] & time [s] & sef time [s] & calls\\
\hline
$Landscape::Update()$ & 50.30 & 58.25 & 58.25 & 1250\\
\hline
$Array2D::operator()$& 15.18 & 75.83& 17.58 & 4806406416\\
\hline
$std::vector::operator[]$&13.72& 91.71 &15.89 & 511439120\\
\hline
\end{tabular}
\end{center}
\end{table}

Run time for this modified implementation was 27.123 seconds which means an improvement of 43\% over the implementation with bounds checking.
While bounds checking was included in the initial version of $Array2D<T>::operator()$ as a precaution, once the program had been demonstrated to work correctly, the considerable performance improvement achieved by removing it was considered to be worthwhile.\\
The reference documentation for class $std::vector$ states that element access with bounds checking is provided by the function $std::vector<T>::at()$.  A timing run was performed with $std::vector<T>::operator[]$ replaced with $std::vector<T>::at()$, resulting in a run time of 35.127 seconds.
This was an improvement of 26\% over the initial implementation but still considerably slower than with no bounds checking at all. 
All three options were left in the code in $Landscape::Update()$ with the unused options commented out.\\

A second optimisation became apparent on examination of the for loops in $Landscape::Update()$. The code was iterating over the x-coordinate of a cell, then over its y-coordinate.
While this was the intuitive order, it caused the code to stride through the array of cells by a distance proportional to the width of the landscape array instead of accessing the next cell in memory each time.
The order of the iterations was swapped, and another profiling and timing run were performed leading to output shown in tab.(\ref{ profile 3})


\begin{table}
\caption{second profile.}
\label{tab: profile 3}
 \begin{center}
\begin{tabular}{|c|c|c|c|c|}
\hline
name & time [\%] & time [s] & sef time [s] & calls\\
\hline
$Landscape::Update()$ & 48.54 & 59.65 & 59.65 &1250 \\
\hline
$Array2D::operator()$& 17.45& 81.10& 21.45 &4806406416 \\
\hline
$std::vector::operator[]$& 12.34& 96.27 &15.17 &511439120\\
\hline
\end{tabular}
\end{center}
\end{table}

Run time with the order of iterations swapped was 12.366 seconds, a reduction of 54\% from the previous implementation and a reduction of 74\% from the original unoptimised implementation.\\

Finally, another optimisation was considered: making every landscape cell store pointers to its neighbours to the north, south, east, and west, so that function $Landscape::Update()$ could access each cell's neighbours without calling $Array2D::operator()$ four times.
Neighbours would be stored during initialisation of the array, meaning the array accesses would be done just once instead of once per update, reducing the number of memory accesses to non-local parts of the array.
However, there was insufficient time for this change to be implemented and tested.\\

It would also have been interesting to see if any difference in performance could be found between runs on landscapes with different arrangements of land and water due to branch prediction as the Landscape class verifies that cell is land before performing the calculation for that cell.
A landscape consisting entirely of islands of one cell would cause the branch prediction to be wrong most if not all of the time, which would presumably reduce performance compared to a landscape with the same number of land cells arranged at the top or bottom of the landscape, corresponding to the beginning or end of the landscape array.




\subection{Performance of the optimised code}
\label{performance of the optimied code}

TODO: add text 

\begin{table}
\caption{Timing for increasing landscape sizes}
\label{tab: Size timing}
 \begin{center}
\begin{tabular}{|c|c|c|}
\hline
Ladscape Size & run time [s] & delta run time[s]\\
\hline
200 & 1.8722 & ×\\
\hline
400 & 11.662 & 9.7898\\
\hline
600 & 25.526 & 13.864\\
\hline
800 & 48.755 & 23.229\\
\hline
1000 & 81.310 & 32.555\\
\hline
1200 & 97.364 & 16.054\\
\hline
1400 & 145.020 & 47.656\\
\hline
1600 & 220.000 & 74.98\\
\hline
1800 & 277.366 & 57.366\\
\hline
200 & 346.391 & 69.025\\
\hline
\end{tabular}
\end{center}
\end{table}





% Some brief conclusions 
\section{Conclusions}
\label{Conclusions}

Even thinking of unittests makes the design more split up and testable...
% Ideas for further work or what you would have done if you had more time.
\section{Further Work}
\label{Further Work}

\subsection{Further Work on the Implementation}
Given further time the group would have liked to include a unit test for the timing methods for completeness.
While a function called \texttt{ApplyPopulation} was added to the \texttt{Landscape} class which would allow the passing of a tailored array of pumas and hares to the class, there was insufficient time to implement this as an option within the program. 

\subsection{Further Performance Testing}

As described in Section \ref{Debugging} there was an array-indexing bug which was not apparent until the program was run on a non-square landscape. 
Given more time the group would have liked to investigate the performance of the code for non-square input maps.
Specifically the group could have investigated program duraton while varying one dimension and keeping the other constant. 
Another interesting test would have been to produce two alternative land/water mask files that were simply rotated ninety degrees and investigate any performance degredation dependant on which dimension is largest. 



 




\end{document}

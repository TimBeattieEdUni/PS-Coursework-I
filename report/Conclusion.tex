% Some brief conclusions 
\section{Conclusions}
\label{Conclusions}

The project did not reach the stage where any serious investigation could be performed into the behaviour of populations over time given various starting conditions.  
However, the program's behaviour was sanity-tested by starting with a low density of hares and verifying by visual inspection of the produced .ppm files that the pumas died out and the hare population increased afterwards, but the group did not have time to include this in this report.

The group found that splitting a large task into many small tasks was an efficient way to share work between members.  
This process benefitted greatly from a modular program design which could be divided in this way and from using a programming language designed for modularity and code reuse.  
Using revision control facilitated both distribution of completed work and communication about ideas and work in progress. 
Careful control of what was allowed to enter the repository, enabled by a simple, one-step build and test of all deliverables, ensured that the group always had a known-good product which could be incrementally improved throughout the time available.

The group also found that clear and frequent communication was essential when working in a team. 

Finally, the group learned from first-hand experience that skipping unit testing, regardless of deadline pressure, can cost significantly more time than it saves.


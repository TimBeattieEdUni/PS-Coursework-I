% How you planned your work
\section{Planning}
\label{Planning}

At the group's first meeting it was decided to create a task list with each member of the group suggesting tasks which would be invoved in completing the project.  Each task was then assigned one of the following priorities:

\begin{itemize}
	\item \textit{Essential:} The task must be done in order to consider the project complete.
	\item \textit{Desirable:} Completing the task was not essential but would likely improve the group's mark for the coursework.
	\item \textit{Polish:} The task was above and beyond the requirements for the project.
\end{itemize}

The group agreed to work on only essential tasks until these were all complete, then on desirable tasks, and only then on tasks designated as polish.

Decisions were then made concerning the infrastructure required for the group to work together: the choice of version control system, build tool, and unit testing framework. The group also agreed on a basic coding standard concerning indentation, placement of curly braces, and file names for C++ code.

The overall program design and layout were then discussed (see sec.(\ref{Design})), focussing on the set of classes which would be required and the information these classes would exchange.

Additional tasks were identified, prioritised, and added to the task list throughout the duration of the project.


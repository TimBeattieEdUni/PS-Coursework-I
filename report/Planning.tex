% How you planned your work
\section{Planning}
\label{Planning}
At the first meeting it was decided to set up version control as detailed in sec.\ref{Revision Control}, and to create a tasklist.

The tasklist was structured along the lines described in the \textit{starting your project} section of the courserwork description; each team member suggested specific tasks relevant to the various sections and these were marked as \textit{essential}, i.e. needed for a basic implementation of the project, or \textit{polish}.

After the tasklist was established we agreed on a coding standard, build system and test framework, see sec.(\ref{Build Tools}) and sec.(\ref{Testing}), which was subsequently setup.
The overall program design and layout was then discussed, see sec.(\ref{Design}), which focussed on the structure of the classes and their Unit Tests and the code required to implement them.
With the essential part of the Tasklist therefore implemented the focus was planned to be shifted to the \tasktt{polish} tasks starting with profiling and optimisation (see sec.(\ref{Performance Tests and Analysis}).

However, in the real execution this was followed only up to the actual coding of the unit tests as profiling and optimisation was performed before careful unit tests were in place. 
